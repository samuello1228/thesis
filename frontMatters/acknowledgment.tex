%*******************************************************
% Acknowledgments
%*******************************************************

\addcontentsline{toc}{chapter}{Acknowledgments}
\chapter*{Acknowledgments}

I would like to take this opportunity to thank people who helped me do the research.

I would like to thank my supervisor Dr. Tu Yanjun for giving me the opportunity to do the research, for her support on my research and for her guidance and suggestions on my analysis.

I would like to thank Dr. Zhang Dongliang.
He helped me get familiar with the linux software: the linux shell, the text editor (vim) and the revision control system (svn).
He taught me how to use the ATLAS software: the event analysis software (ROOT and RootCore), the data acquisition software(dq2 and rucio) and the distributed computing management (grid and panda).
I learnt from him many other experimental skill.
He also helped our group write the multiLepSearch package to generate the ntuple.

I would like to thank Dr. Daniela Katherinne Paredes Hernandez.
She gave a lot of suggestions on the analysis.
She designed the validation region and helped me do the analysis.

I would like to thank Dr. Jeanette Miriam Lorenz from the Ludwig Maximilian University of Munich.
She provided the configuration file for HistFitter and organized various tasks during collaboration.

I would like to thank Dr. Peter Tornambe from Albert the Ludwig University of Freiburg.
He helped me compare my fake lepton background measurements with his measurements by helping me generate the Monte Carlo ntuples.

I would like to thank Dr. Daniela Maria Kock from the Ludwig Maximilian University of Munich.
She provided me the ntuples for systematics study.

The study of the systematic uncertainties is not done by me, therefore it is not included here.
The following are my contributions to the public paper \cite{Wh} \cite{WhSS}.
\begin{enumerate}
\item An alternative method for the SR optimization are used as a cross-check for the signal regions optimized by Daniela Maria Kock. I also suggested that she can use $m_{T2}$ for optimization.
\item I contribute to the ntuple production for the charge flip background. I also maintain the framework of the charge flip estimation.
\item I provide the cross-check to the work by Peter Tornambe (fake lepton background) and Daniela Paredes (validation region).
\end{enumerate}
