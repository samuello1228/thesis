%*******************************************************
% Abstract
%*******************************************************
\phantomsection
\addcontentsline{toc}{chapter}{Abstract}

\begin{center}

Abstract of thesis entitled\\

\bigskip

    \huge\textbf{Search for chargino and neutralino production in final states with two same-sign leptons, jets and missing transverse momentum at $\sqrt{s} = 13$ TeV with the ATLAS detector} \\

    \bigskip

    {\normalsize Submitted by}\\

\bigskip

    \Large{\textbf{Cheuk Yee LO}}\\

\bigskip

{\normalsize
for the degree of Doctor of Philosophy\\
at The University of Hong Kong\\
in November 2018\\}

\end{center}

\bigskip

The Standard Model in particle physics has successfully explained almost all experimental results in the microscopic scale with high accuracy.
However, the nature of the dark matter and the hierarchy problem of Higgs mass are still the unanswered questions.
Supersymmetry (SUSY) is one of the most promising theories beyond the Standard Model, that might answer these questions.
In the recent searches for supersymmetric particles that are involved in strong interaction, their masses are above 1 TeV.
This might suggest that the pair production of electroweak gauginos is a dominant SUSY production process at the Large Hadron Collider (LHC).
Also, the recent upgrade that the center-of-mass energy of the proton-proton collisions $\sqrt{s}$ has increased to 13 TeV, opened a new phase of exploration for SUSY.
% word count: 122

In this thesis, a search is presented for the electroweak pair production of a chargino and a neutralino ($p + p \rightarrow \tilde{\chi}_1^\pm + \tilde{\chi}_2^0$),
where the chargino decays to the lightest neutralino and a W boson ($\tilde{\chi}_1^\pm \rightarrow \tilde{\chi}_1^0 + W$),
and the second lightest neutralino decays to the lightest neutralino and a Standard Model like Higgs boson with mass 125 GeV ($\tilde{\chi}_2^0 \rightarrow \tilde{\chi}_1^0 + h$).
The final state with two same-sign leptons, jets and missing transverse momentum are considered in this search.
The two leptons come from the leptonically decay of the W boson and the Higgs boson with the decay modes of  $h \rightarrow WW$, $h \rightarrow \tau \tau$ or $h \rightarrow ZZ$.
This analysis is based on the proton-proton collision data delivered by the LHC at $\sqrt{s}$ = 13 TeV with the ATLAS detector.
The integrated luminosity of data is 36.1 fb$^{-1}$.
% word count: 121

%The results of this analysis are combined with the results of the search for final state with 3 leptons.
The exclusion limits for the masses of $\tilde{\chi}_1^\pm$ and $\tilde{\chi}_2^0$ are extended up to 245 GeV, while the exclusion limits for the mass of $\tilde{\chi}_1^0$ are extended up to 40 GeV, with 95\% confidence level, in the context of a simplified supersymmetric model.
% word count: 43

\bigskip

\begin{center}

\rule{6cm}{0.025cm}\\
{\slshape An abstract of exactly 286 words}

\end{center}
