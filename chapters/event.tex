%************************************************
\chapter{Dataset inputs and event selection}
\label{ch:event}
%************************************************

\section{Dataset inputs}
This chapter describes the dataset used in this analysis.
The dataset contains the data samples and Monte-Carlo(MC) simulated sample.
All dataset are SUSY2 DxAOD derivations, which aim for 2 or 3 leptons search.

\subsection{Data samples}
We use 2015 (periods D-H and J) and 2016 (periods A-L, I, K and L) $pp$-collisions data samples, at $\sqrt s=13$ TeV.
Only events in good condition are used, where LHC beams were stable and all ATLAS detectors were in good state.
These good events are summarised in the Good Run Lists.
The two Good Run Lists (GRL) in 2015 and 2016 data are shown in the section \ref{sec:List_Data}.
The integrated luminosities in 2015 and 2106 are $3.21$ fb$^{-1}$ and $32.86$ fb$^{-1}$ respectively, with relative error 2.1\%.
The list of data samples used in this analysis is shown in the section \ref{sec:List_Data}.

\subsection{MC samples}
\subsubsection{SM background}
All MC samples are mc15c samples with offline release 20.7.
All the background MC samples used in this analysis for each process are shown in the section \ref{sec:MCBG} in appendix.
Each sample has its cross section, k-factor, generator efficiency and its equivalent integrated luminosity.
Some samples may overlap with each other.

\paragraph{$\bf t\bar{t}$ and single top}
The simulated events are generated by the {\sc POWHEG} generator, and the CT10 PDF set is used.
{\sc Pythia6} is also used for the parton shower model, with the {\sc Perugia} 2012 tune.
The mass of the top quark is assumed to be 172.5 GeV.
The $t\bar{t}$ samples are normalized to the next-to-next-to-leading order of cross section, while the single top samples are normalized to the next-to-leading order of cross section.

\paragraph{\bf W+jets and Z+jets}
The simulated events are generated by the {\sc SHERPA} v2.2.1.
The matrix elements are calculated at the next-to-leading order for up to two partons, and at the leading order for up to four partons, by using the {\sc Comix} and {\sc OpenLoops} generators.
The samples are normalized to the next-to-next-to-leading order QCD cross section.
The files are separated according to the $p_T$ of the vector boson and the presence of $b$-jet and $c$-jets.

\paragraph{\bf Diboson}
The processes with four charged leptons ($\ell \ell \ell \ell$), three charged leptons and one neutrino ($\ell \ell \ell \nu$), and two charged leptons and two neutrinos ($\ell \ell \nu \nu$) are simulated by the {\sc SHERPA} v2.2.1 generator.
Diboson $WW$, $WZ$ and $ZZ$ processes with four or six electroweak vertices are also used.

\paragraph{\bf Triboson}
The triboson processes $WWW$, $WWZ$, $WZZ$ and $ZZZ$ with up to six charged leptons are simulated by the {\sc SHERPA} v2.2.1 generator.

\paragraph{\bf ttV}
The processes $ttW$, $ttZ$, $ttWW$ and $ttWZ$ are simulated by {\sc MadGraph} v2.2.2 at the leading-order, with {\sc Pythia} for the parton shower model.

\paragraph{\bf Higgs}
The $WH$ and $ZH$ processes are generated by using {\sc Pythia} 8 generator, and the {\sc A14} set of tuned parameters is used together with the {\sc NNPDF23LO} PDF set.
The $ttH$ processes are generated by using {\sc Mcatnlo} generator, interfaced with {\sc Herwigpp}.
The CT10 PDF tuning is used along with the CTEQ6L1-UE-EE-5 tuning of parton shower.

\subsubsection{Signal}
The signal MC samples simulate the signal process $\tilde{\chi}_1^\pm \tilde{\chi}_2^0 \rightarrow W(\ell\nu)h$.
They are generated by the {\sc MadGraph} v2.2.3, calculated at the leading-order matrix elements with up to two extra partons.
{\sc Pythia} version 8.186 and the A14 tune are also used for the modelling of the SUSY decay chain, parton showering and hadronization.
Parton luminosities are provided by the NNPDF23LO PDF set.
Table \ref{table:signal} shows the list of signal samples used in this analysis, with different hypothesized masses point ($m_{\tilde{\chi}_1^\pm , \tilde{\chi}_2^0}$, $m_{\tilde{\chi}_1^0}$).
These signal samples have been applied a selection that at least 2 leptons with $p_{T} >$ 7 GeV is required.
The efficiencies due to this selection are applied and also shown in the table.

\section{Pre-selection and event cleaning}
\label{event_cleaning}
The following pre-selections on the events are applied to reject background which did not come from the proton-proton collision and to ensure that the detector was working properly.

\begin{itemize}
\item \textbf{Good Run List} The events need to pass the good run list. (For data only)
\item \textbf{LAr/Tile/SCT error} Events with data integrity errors in the SCT detector and the LAr and Tile calorimeter are removed. (For data only)
\item \textbf{Primary Vertex} The events need to have a primary vertex, which is defined as the one with the largest $\sum p_{T}^{2}$ of tracks, and has at least two tracks.
\item \textbf{Cosmic Muon Veto} The events with cosmic muons need to be removed. The track of cosmic muon is identified by large impact parameters with respect to the primary vertex, with the condition that $|z_{0}^{PV}|>1$ mm or $|d_{0}^{PV}|>0.2$ mm.
\item \textbf{Bad Muon Veto} The events with a bad muon that does not come from the proton-proton collision need to be removed. The bad muon is identified by a large relative error in the ratio of electric charge to momentum ($q/p$), with the condition that $\sigma(q/p) / |q/p| > 0.2$, suggested by the recommendation of the Muon CP group. \cite{muons}
\item \textbf{Bad Jet Veto} The events with bad jets that does not come from the proton-proton collision need to be removed. A jet with $p_{T}<$ 20 GeV or with the ``LooseBad" quality by the recommendation of the Jet/$E_T^{\text{miss}}$ group is identified as a bad jet.
\item \textbf{Trigger Selection} The events need to pass at least one trigger in the trigger list, described in section \ref{sec:trigger}.
\item \textbf{Exactly 2 baseline leptons} The events which have exactly 2 baseline leptons are selected. The definition of baseline electron and muon are described in section \ref{sec:object}. ``The two leptons" mentioned in the later chapters refer to these 2 baseline leptons. These two leptons are indexed in the descending order by their $p_T$. The lepton with larger $p_T$ is called the leading lepton ($\ell_1$), and the lepton with smaller $p_T$ is called the sub-leading lepton($\ell_2$).
\end{itemize}

\section{Trigger strategy}
\label{sec:trigger}
The time-spacing between two adjacent bunches is 25 ns, and equivalently the frequency is 40 MHz.
Because not all the collisions will be our interested events, and it is also infeasible to store all the events generated by the LHC to the permanent storage, the trigger strategy is used.
The trigger system accepts and rejects the events immediately after the data is taken.
The Level 1 trigger system filters the events from 40 MHz to 100 kHz.
The High Level trigger (HLT) system uses the output from the Level 1 trigger system, and further filters the events from 100 kHz to 1 kHz.

In our analysis, the single lepton trigger and di-lepton trigger was used. Table \ref{tab:single_trigger} and \ref{tab:dilepton_trigger} show the list of triggers used in this analysis.

\begin{table}[htbp]
\begin{center}
\begin{tabular}{|c|c|c|}
\hline
& \textbf{Single electron} & \textbf{Single muon}\\
\hline
\hline
2015 & \texttt{HLT\_e24\_lhmedium\_L1EM20VH} & \texttt{HLT\_mu20\_iloose\_L1MU15}\\
& \texttt{HLT\_e60\_lhmedium}           & \texttt{HLT\_mu40}\\
& \texttt{HLT\_e120\_lhloose}           & \\
\hline
2016 & \texttt{HLT\_e26\_lhtight\_nod0\_ivarloose} & \texttt{HLT\_mu26\_imedium}\\
& \texttt{HLT\_e60\_lhmedium\_nod0}          & \texttt{HLT\_mu50}\\
& \texttt{HLT\_e140\_lhloose\_nod0}          & \\
\hline
\end{tabular}
\end{center}
\caption{List of the single lepton triggers used in this analysis.}
\label{tab:single_trigger}
\end{table}

\begin{table}[htbp]
\begin{center}
\begin{tabular}{|c|c|c|c|}
\hline
& \textbf{Di-electron} & \textbf{Di-muon} & \textbf{Electron-muon}\\
\hline
\hline
2015 & \texttt{HLT\_2e12\_lhloose\_L12EM10VH} & \texttt{HLT\_mu18\_mu8noL1} & \texttt{HLT\_e17\_lhloose\_mu14}\\
     &                                        &                             & \texttt{HLT\_e7\_lhmedium\_mu24}\\
\hline
2016 & \texttt{HLT\_2e17\_lhvloose\_nod0}     & \texttt{HLT\_mu22\_mu8noL1} & \texttt{HLT\_e17\_lhloose\_nod0\_mu14}\\
     &                                        &                             & \texttt{HLT\_e7\_lhmedium\_nod0\_mu24}\\
\hline
\end{tabular}
\end{center}
\caption{List of the di-lepton triggers used in this analysis.}
\label{tab:dilepton_trigger}
\end{table}

\section{Object definitions}
\label{sec:object}
The object definitions are based on \texttt{SUSYTools-00-08-60} and analysis release \texttt{Base,2.4.31}, and their associated performance packages.

\subsection{Elections}
Electrons are reconstructed by using the recommendations from the egamma group and need to be inside the region $|\eta^{\text{cluster}}|<2.47$.
The baseline electrons are identified by the \texttt{LooseAndBLayerLLH} quality criterion and have $p_T>10$ GeV.
The signal electrons must be baseline electrons and satisfy additional criteria.
At the signal level, the electron must satisfy the \texttt{MediumLLH} quality criterion and has $p_T>25$ GeV.
The working point for the isolation cut is \texttt{FixedCutTight}.
The requirement for the impact parameter is $|z_0 \cdot \sin (\theta)|< 0.5$ mm and $|d_0/\sigma(d_0)|<5$, recommended by the Tracking CP group.
To reduce the charge flip background, \texttt{ChargeIDSelector} is used with the working point \texttt{Medium} at 97\% efficiency.
The selections for baseline and signal electrons are summarised in table \ref{tab:lepdef}.

\subsection{Muons}
Muons are reconstructed by using the recommendation from the MCP group \cite{muons} and requiring $|\eta|<2.4$.
The baseline muons are identified by the \texttt{Medium} quality criterion and have $p_T>10$ GeV.
The signal muons must be baseline muons and satisfy additional criteria.
The additional criteria are $p_T>25$ GeV and isolation cut with the working point \texttt{GradientLoose}.
The requirement for the impact parameter is $|z_0 \cdot \sin (\theta)|< 0.5$ mm and $|d_0/\sigma(d_0)|<3$, recommended by the Tracking CP group.
The selections for baseline and signal muons are summarised in table \ref{tab:lepdef}.

\begin{table}[htbp]
\begin{center}
\begin{tabular}{|l|c|c|}
\hline
& \textbf{Baseline Electron} & \textbf{Baseline Muon} \\
\hline
\hline
Acceptance     & $p_T > 10$ GeV, $|\eta^{\text{cluster}}| < 2.47$  & $p_T > 10$ GeV, $|\eta| < 2.4$ \\
\hline
Quality & LooseAndBLayerLLH & Medium \\
\hline
\hline
& \textbf{Signal Electron} & \textbf{Signal Muon} \\
\hline
\hline
Acceptance & $p_T > 25$GeV & $p_T > 25$ GeV \\
\hline
Quality & MediumLLH & Medium \\
\hline
Isolation Cut  & \texttt{FixedCutTight} & \texttt{GradientLoose} \\
\hline
Impact parameter & $|z_0 \cdot \sin (\theta)|< 0.5$ mm   & $|z_0 \cdot \sin (\theta)|< 0.5$ mm \\
& $|d_0/\sigma(d_0)|<5$ & $|d_0/\sigma(d_0)| < 3$\\
\hline
ChargeIDSelector & Medium at 97\% efficiency & - \\
\hline
\end{tabular}
\end{center}
\caption{Summary of the electron and muon selection criteria. The signal selection requirements are applied on top of the baseline criteria.}
\label{tab:lepdef}
\end{table}

\subsection{Jets}
The baseline jets are reconstructed by the anti-$k_t$ jet algorithm with the distance parameter $D = 0.4$.
Each baseline jet must have $p_T > 20$ GeV and $|\eta| < 2.8$.
The signal jets are selected on top of the baseline jet, with additional criteria.
The signal jets need to further satisfy the Jet Vertex Tagger (JVT) cut that JVT$> 0.59$ if the jets have $p_T < 60$ GeV and $|\eta| < 2.4$.
The b-jets are signal jets with b-tag, by using the MV2c10 b-tagging algorithm with \texttt{FixedCut} working point which has b-jet efficiency 77\%.
The selections of jets are summarised in table \ref{tab:jetsdef}.

\begin{table}[htbp]
\begin{center}
\begin{tabular}{|l|c|}
\hline
\hline
& \textbf{Baseline Jet} \\
\hline
\hline
Collection     & AntiKt4EMTopo \\
\hline
Acceptance     & $p_T > 20$ GeV, $|\eta| < 2.8$ \\
\hline
\hline
& \textbf{Signal Jet} \\
\hline
\hline
Acceptance     & $p_T > 20$ GeV, $|\eta | < 2.8$ \\
\hline
Jet vertex tagger  &  \texttt{Medium} working point\\
&  JVT $> 0.59$ for $p_T < 60$ GeV and $|\eta| < 2.4$\\
\hline
\hline
& \textbf{B-Jet} \\
\hline
\hline
Acceptance     & $p_T > 20$ GeV, $|\eta| < 2.4$ \\
\hline
$b-$tagging algorithm &  MV2c10 algorithm \\
Working point & \texttt{FixedCut} with efficiency 77\% \\
\hline
\end{tabular}
\end{center}
\caption{Summary of the jet selection criteria.}
\label{tab:jetsdef}
\end{table}

\subsection{Missing transverse momentum}
Based on the conservation of transverse momentum, the total transverse momentum of the missing particles,  which were not detected by the detector, can be estimated by the total transverse momentum of particles which can be detected.
The missing transverse momentum (${\bf p}_T^{\text{miss}}$) is defined by the negative of the sum of transverse momentum of all electrons, muons, photons, jets and all other tracks associated with the primary vertex.
The calibrated electrons, muons, photons and jet objects are used as the inputs.
This missing transverse momentum can estimate the total transverse momentum of the missing neutrinos and hypothetical neutralinos.
The Missing transverse energy ($E_{T}^{\text{miss}}$) is defined by the magnitude of the missing transverse momentum ${\bf p}_T^{\text{miss}}$.

\subsection{Overlap removal}
\label{sec:overlapRemoval}
The overlap removal (OR) is performed with the baseline objects (electrons, muons and jets) and follows the default prescription provided in the \texttt{SUSYTools}.
The objects are removed in the following order.
\begin{enumerate}
\item If a jet is within $\Delta R= \sqrt{(\Delta \eta)^2 + (\Delta \phi)^2}=0.2$ of an electron:
\begin{itemize}
\item If the jet is not $b-$tagged, then the jet is removed. It mostly originates from the calorimeter energy deposits by the electron shower.
\item If the jet is $b-$tagged, then the electron is removed. It is more likely that it results from the semi-leptonic decays of $b-$quarks.
\end{itemize}
\item Electrons within $\Delta R=0.4$ of a jet are removed, in order to suppress electrons from semi-leptonic decays of $c$- and $b$-hadrons.
\item Muons within $\Delta R=0.4$ of a jet are removed, in order to suppress muons from semi-leptonic decays of $c$- and $b$-hadrons.
\item Any calo-tagged muons sharing the same ID track with an electron are removed.
\item Any electrons sharing the same ID track with the remaining muons are removed.
\end{enumerate}
