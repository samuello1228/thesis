%************************************************
\chapter{Dataset inputs and event selection}
\label{ch:event}
%************************************************

\section{Dataset inputs}
This chapter describes the dataset used in this analysis.
The dataset contains the data samples and Monte-Carlo(MC) simulated sample.
All dataset are SUSY2 DxAOD derivations, which aim for 2 or 3 leptons search.

\subsection{Data samples}
We use 2015 (periods D-H and J) and 2016 (period A-L, I, K and L) $pp$-collisions data samples, at $\sqrt s=13$ TeV.
Only events with good condition are used, where LHC beams were stable and all ATLAS detectors were in good state.
The following two Good Run Lists (GRL) describe what events is used in 2015 and 2016 data.
\begin{itemize}
\item {\ttfamily\scriptsize data15\_13TeV.periodAllYear\_DetStatus-v79-repro20-02\_DQDefects-00-02-02\_PHYS\_StandardGRL\_All\_Good\_25ns.xml} for 2015 data.
\item {\ttfamily\scriptsize data16\_13TeV.periodAllYear\_DetStatus-v89-pro21-01\_DQDefects-00-02-04\_PHYS\_StandardGRL\_All\_Good\_25ns.xml} for 2016 data.
\end{itemize}
The integrated luminosities in 2015 and 2106 are $3.21$ fb$^{-1}$ and $32.86$ fb$^{-1}$ respectively, with relative error 2.1\%.
The following are the list of data samples used.
\scriptsize
\begin{verbatim}
data15_13TeV.periodD.physics_Main.PhysCont.DAOD_SUSY2.grp15_v02_p2950
data15_13TeV.periodE.physics_Main.PhysCont.DAOD_SUSY2.grp15_v02_p2950
data15_13TeV.periodF.physics_Main.PhysCont.DAOD_SUSY2.grp15_v02_p2950
data15_13TeV.periodG.physics_Main.PhysCont.DAOD_SUSY2.grp15_v02_p2950
data15_13TeV.periodH.physics_Main.PhysCont.DAOD_SUSY2.grp15_v02_p2950
data15_13TeV.periodJ.physics_Main.PhysCont.DAOD_SUSY2.grp15_v02_p2950
data16_13TeV.periodA.physics_Main.PhysCont.DAOD_SUSY2.grp16_v02_p2950
data16_13TeV.periodB.physics_Main.PhysCont.DAOD_SUSY2.grp16_v02_p2950
data16_13TeV.periodC.physics_Main.PhysCont.DAOD_SUSY2.grp16_v02_p2950
data16_13TeV.periodD.physics_Main.PhysCont.DAOD_SUSY2.grp16_v02_p2950
data16_13TeV.periodE.physics_Main.PhysCont.DAOD_SUSY2.grp16_v02_p2950
data16_13TeV.periodF.physics_Main.PhysCont.DAOD_SUSY2.grp16_v02_p2950
data16_13TeV.periodG.physics_Main.PhysCont.DAOD_SUSY2.grp16_v02_p2950
data16_13TeV.periodI.physics_Main.PhysCont.DAOD_SUSY2.grp16_v02_p2950
data16_13TeV.periodK.physics_Main.PhysCont.DAOD_SUSY2.grp16_v02_p2950
data16_13TeV.periodL.physics_Main.PhysCont.DAOD_SUSY2.grp16_v02_p2950
\end{verbatim}
\normalsize

\subsection{MC samples}
\subsubsection{SM background}
All MC samples are mc15c samples with offline release 20.7.
All the background MC samples used in this analysis for each processes are shown the section \ref{sec:MCBG} in appendix.
Each samples has its cross section, k-factor, generator efficiency and their equivalent integrated luminosity.
Some samples may overlap with each other.

\paragraph{$\bf t\bar{t}$ and single top}
The simulated events are generated by the {\sc POWHEG} generator, and the CT10 PDF set is used.
{\sc Pythia6} is also used for the parton shower model, with the {\sc Perugia} 2012 tune.
The mass of the top quark is assumed to be 172.5 GeV.
The $t\bar{t}$ samples are normalized to the next-to-next-to-leading order of cross section, while the single top samples are normalized to the next-to-leading order of cross section.

\paragraph{\bf W+jets and Z+jets}
The simulated events are generated by the {\sc SHERPA} v2.2.1.
The matrix elements are calculated at the next-to-leading order for up to two partons, and at the leading order for up to four partons, by using the {\sc Comix} and {\sc OpenLoops} generators.
The samples are normalized to the next-to-next-to-leading order QCD cross section.
The files are seperated according to the $p_T$ of the vector boson and the presence of $b$-jet and $c$-jets.

\paragraph{\bf Diboson}
The processes with four charged leptons ($\ell \ell \ell \ell$), three charged leptons and one neutrino ($\ell \ell \ell \nu$), and two charged leptons and two neutrinos ($\ell \ell \nu \nu$) are simulated by the {\sc SHERPA} v2.2.1 generator.
Diboson $WW$, $WZ$ and $ZZ$ processes with four or six electroweak vertices are also used.

\paragraph{\bf Triboson}
The triboson processes $WWW$, $WWZ$, $WZZ$ and $ZZZ$ with up to six charged leptons are simulated by the {\sc SHERPA} v2.2.1 generator.

\paragraph{\bf tt+boson}
The processes $ttW$, $ttZ$, $ttWW$ and $ttWZ$ are simulated by {\sc MadGraph} v2.2.2 at the leading-order, with {\sc Pythia} for the parton shower model.

\paragraph{\bf Higgs}
The $WH$ and $ZH$ processes are generated by using {\sc Pythia} 8 generator, and the {\sc A14} set of tuned parameters is used together with the {\sc NNPDF23LO} PDF set.
The $ttH$ processes are generated by using {\sc Mcatnlo} generator, interfaced with {\sc Herwigpp}.
The CT10 PDF tuning is used along with the CTEQ6L1-UE-EE-5 tuning of parton shower.

\subsubsection{Signal}
The siganl MC samples simulate the signal process $\tilde{\chi}_1^\pm \tilde{\chi}_2^0 \rightarrow W(\ell\nu)h$.
They are generated by the {\sc MadGraph} v2.2.3, calculated at the leading-order matrix elements with up to two extra partons.
{\sc Pythia} version 8.186 and the A14 tune are also used for the modelling of the SUSY decay chain, parton showering and hadronisation.
Parton luminosities are provided by the NNPDF23LO PDF set.
Table \ref{table:signal} shows the list of signal samples used in this analysis, with different hypothesized masses of $\tilde{\chi}^{\pm}_{1}$/$\tilde{\chi}^{0}_{2}$ and $\tilde{\chi}^{0}_{1}$.
These signal samples have been applied a selection that at least 2 leptons with $p_{T} >$ 7 GeV is required.
The efficiencies due to this selection are applied and also shown in the table.

\section{Pre-selection and event cleaning}
The following pre-selections on the events are applied to reject background which did not come from the proton-proton collision and to ensure that the detector was working properly.

\begin{itemize}
\item \textbf{Good Run List} The events need to pass the good run list. (For data only)
\item \textbf{LAr/Tile/SCT error} Events with data integrity errors in the SCT detector and the LAr and Tile calorimeter are removed. (For data only)
\item \textbf{Primary Vertex} The events need to have a primary vertex, which is defined as the one with the largest $\sum p_{T}^{2}$ of tracks, and has at least two tracks.
\item \textbf{Cosmic Muon Veto} The events with cosmic muons need to be removed. The track of cosmic muon is identified by large impact parameters with respect to the primary vertex, with the condition that $z_{0}^{PV}>1$ mm or $d_{0}^{PV}>0.2$ mm.
\item \textbf{Bad Muon Veto} The events with bad muons that does not come from the proton-proton collision need to be removed. The bad muon is identified by a large relative error in the ratio of electric charge to momentum ($q/p$), with the condition that $\sigma(q/p) / |q/p| > 0.2$.
\item \textbf{Bad Jet Veto} The events with bad jets that does not come from the proton-proton collision need to be removed. A jet with $p_{T}<$ 20 GeV or with the ``LooseBad" quality by the Jet/Etmiss group is identified as a bad jet.
\end{itemize}
