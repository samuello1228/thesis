%************************************************
\chapter{Experimental Setup}
\label{ch:detector}
%************************************************

\section{Introduction}
\label{sec:detector_introduction}

Our experimental data was collected from the ATLAS particle detector in the Large Hadron Collider (LHC).
The following section will introduce LHC and the ATLAS particle detector.

\section{The Large Hadron Collider}
\label{sec:detector_LHC}

The Large Hadron Collider (LHC) was built in the border between France and Switzerland by the European Organization for Nuclear Research (CERN).
It is a circular particle collider under the ground with circumference 27 km.
Two beams of protons will be accelerated in opposite directions, and then these two beams will collide with each other at the collision point.
The energy of each beam is 6.5 TeV, and hence the center-of-mass energy of the two beams $\sqrt{s}$ is 13 TeV, which is the energy used in this experiment.
This energy is equivalent to the speed that the beam will circulate the ring 11,245 times per second.
Figure \ref{fig:detector_LHC_accelerator_complex} shows the schematic diagram of the CERN accelerator complex, which contains a series of accelerators, from low energy to high energy.
The dark blue big circle in figure \ref{fig:detector_LHC_accelerator_complex} represents the LHC, on which there are 4 particle detectors at 4 different yellow points: ATLAS, CMS, LHCb and ALICE.

\begin{figure}
\centering
\includegraphics[width=\textwidth]{data/photo/accelerator_complex.png}
\caption{The schematic diagram of the CERN accelerator complex, which shows a series of accelerators and facilities. \cite{complex}}
\label{fig:detector_LHC_accelerator_complex}
\end{figure}

Before the beam is injected into LHC, the protons need to be accelerated by a series of accelerators.
The journey of the protons starts from a tank of hydrogen gas.
The proton and the electron are separated by a electric field.
The protons are then accelerated to 50 MeV by Linac2, which is a linear accelerator.
The beam is then injected to the second accelerator called the Proton Synchrotron Booster (PSB), which accelerates the beam to 1.4 GeV.
The beam is then injected to the third accelerator called the Proton Synchrotron (PS), which pushes the beam to 25 GeV.
The beam is then injected to the fourth accelerator called the Super Proton Synchrotron (SPS), which further pushes the beam to 450 GeV.
Finally, the beam is injected to the two beam pipes of the LHC.
One of the beam moves in clockwise direction, while another beam moves in anti-clockwise direction.
\cite{accelerator}

The circular path of the proton beam is maintained by many superconducting electromagnets along the LHC tunnel.
There are 1232 main magnetic dipoles, and each of them generates a large magnetic field of 8.3 T.
In order to generate such a high magnetic field, the coils need to have very high currect of 11,080 A, and hence supercoducting coil need to be used, to reduce the heat loss due to the electrical resistance.
The coil is made up of niobium-titanium (NbTi).
To reach the condition for supercoductivity, the electromagnets operate at a very low temperature of 1.9 K.
\cite{supermagnet,cryogenics}


